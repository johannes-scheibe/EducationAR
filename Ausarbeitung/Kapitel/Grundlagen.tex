\chapter{Grundlagen}\label{chapter:Grundlagen}
Dieses Kapitel behandelt die Grundlagen, die für die Realisierung des Prototyps notwendig sind. Dabei werden Methoden und Eigenschaften der Augmented Reality erläutert, sowie ein Überblick über OpenGL gegeben.

\section{Augmented Reality}\label{AR}
Augmented Reality  beschreibt das Einbinden und Visualisieren digitaler, computergenerierte Objekte in der realen Welt. Dabei wird meist  eine möglichst realistische Darstellung dieser Objekte angestrebt.\\
In der Literatur AR oftmals als eine Vereinigung oder Überlagerung der realen Welt mit virtuellen Einblendungen beschrieben \citep{}.


 
\subsection{Marker}

\section{OpenGL}\label{OpenGL}
