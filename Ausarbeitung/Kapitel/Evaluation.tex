\chapter{Evaluation}\label{chapter:evaluation}
In diesem Kapitel werden die Ergebnisse der Arbeit evaluiert. Dabei wird zum einen auf den entwickelten Prototypen eingegangen und zum anderen das Konzept und die Anwendung von Augmented Reality im Bildungsbereich evaluiert.

\section{Evaluation des Prototyps}
Dieses Kapitel dient dazu den Prototypen anhand der in Kapitel \ref{chapter:anforderungsanalyse} gestellten Anforderungen zu evaluieren. Dazu wird zunächst das Testverfahren, das parallel zu der Implementierung genutzt wurde beschrieben und im Anschluss der finale Prototyp evaluiert.

\subsection{Bewertung des finalen Prototyps}




\subsection{Zusammenfassung}\label{sec:Zusammenfassung}
\todo{Am Ende schreiben}
- Belichtung nicht einfach zu testen
- starke Belichtungswechsel teilweise kurze Aussetzer

