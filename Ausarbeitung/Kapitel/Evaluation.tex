\chapter{Evaluation}\label{chapter:evaluation}
In diesem Kapitel werden die Ergebnisse der Arbeit evaluiert. \\
Dabei werden zwei Ziele verfolgt. Zum einen soll der Prototyp anhand der gestellten Anforderungen bewertet werden und zum anderen sollen Probleme und Verbesserungspotentiale des entwickelten Konzepts und der Umsetzung des Prototyps erfasst werden. \\
Dazu wird die Evaluation anhand der Evaluationsziele in eine bewertende Evaluation, bei der die Anwendung auf bestimmte Systemeigenschaften überprüft werden soll, und eine analytische Evaluation, welche die Schwachstellen des Prototyps herausarbeiten soll, unterteilt \citep{hegner:evaluation}.


\section{Bewertende Evaluation}\label{sec:bewertende-evaluation}
Dieses Kapitel dient dazu den Prototypen anhand bestimmter Systemeigenschaften, den in Kapitel \ref{chapter:anforderungsanalyse} gestellten Anforderungen, zu evaluieren. 

\subsection{Vorgehen}
Um den finalen Prototyp auf die gestellten Anforderungen zu überprüfen, wurde dieser auf das Testgerät (ein Huawei P30 Pro) geladen und verschiedenen Testfällen unterzogen, die zum Großteil bereits in Kapitel \ref{sec:tests} beschrieben wurden. Eine genaue Dokumentation von jedem einzelnen Test ist in ..... zu finden. \todo{Verweis auf Testdokumentation}
Im folgenden Abschnitt werden einmal die Ergebnisse des abschließenden Testes zusammengefasst.

\subsection{Ergebnisse}
\subsubsection{Allgemeine Anforderungen}
Insgesamt wurden alle Anforderungen, die sich an die Art der Implementierung richten, umgesetzt.
So läuft die Anwendung auf dem Huawei P30 Pro und wurde dem entsprechend mit Android und Android Studio entwickelt. Auch die Anforderung, dass ein kamera- und markerbasiertes Verfahren verwendet werden soll wurde erfüllt.
\subsubsection{Tracking}
Die Tests haben gezeigt, dass das Tracking stabil mit 60 FPS läuft. Dabei ist es stabil invariant gegenüber verschiedenen Umgebungsveränderungen. Lediglich starke Belichtungswechsel konnten für minimale Aussetzer bei dem Tracking sorgen.

\subsubsection{Modelldarstellung}
Die Modelle, die im OBJ-Format hochgeladen wurden konnten in der Anwendung korrekt mit der richtigen Transformation dargestellt werden und auch die Texturen, die als jpeg oder png-Datei hochgeladen wurden, wurden korrekt angewandt. Bei den Testmodellen wurden sowohl simple selbst erstellte Modelle getestet, als auch sehr komplexe Modelle aus dem Internet mit bis zu 600.000 Faces (Flächen). Alle konnten ohne Problem angezeigt werden, jedoch erhöhen sich die Ladezeiten, je mehr Modelle in die Anwendung geladen wurden und je größer diese sind, dem entsprechend stark. Auch die FPS Anzahl wurde bei vielen gleichzeitig zu rendernden Modellen entsprechend beeinflusst. Diese ist aber erst bei mehreren 100 MB wirklich zu beachten. So sank bei einem Extremtesr die Framerate bei dem parallelen Rendern von drei Modellen mit kombiniert etwa 1.000.000 Faces  (eine DAteigröße von ungefähr 100 MB) und deren Texturdateien, die zusammen ebenfalls eine Größe von fast 100 MB hatten, von den anfänglichen 60 auf 25 FPS. Diese Framerate ist noch nicht bedenklich, da das Tracking zwar weiterhin flüssig war, jedoch sollte diese Auswirkung, besonders mit Blick  auf ältere Geräte, bedacht werden. \\
Da allerdings sein so großer Detailgrad auf einem relativ kleinen Smartphone- oder Tabletdisplay, kaum erkennbar ist, ist eine generelle Vorverarbeitung der Modelle, bei der die Anzahl der Faces(), also der einzelnen Flächen aus denen das Modell besteht, verringert werden, sinnvoll. Dieser Schritt ist sowie so sinnvoll, da die Modelle aus dem Internet oftmals in einer falschen Größe und Rotation vorliegen und kann ohne große Modellierungskenntnisse in kostenlosen Programmen, wie Blender, durchgeführt werden.\\
Als Richtwert, wäre hier ein eine maximale Anzahl von 150.000 Faces und eine maximale Textur Auflösung von 2500x2500 Pixeln denkbar, da ein höherer Detailgrad mit dem bloßen Auge kaum wahrnehmbar. Die Verwendung von speichereffizienten Bildformaten, wie jpeg anstelle von png, ist eine zusätzliche Möglichkeit die Ladezeiten zu verringern.

\subsubsection{Markergenerierung}
Eine weitere Eigenschaft die getestet wurde war die Markergenerierung. Hier erfüllt die Anwendung alle Kriterien und kann insgesamt 4.194.304 Marker generieren. Anhand einer stichprobenartigen Überprüfung (vergleiche Kapitel \ref{sec:test-marker}) konnte festgestellt werden, dass die generierten Marker auch problemlos im Tracking mit der korrekten ID erkannt werden.

\subsubsection{Benutzeroberfäche}
Die Benutzeroberfläche ist simple gehalten und bietet Navigationsmöglichkeiten zu allen wichtigen Funktionen. So sind Ansichten zum Hochladen und Verwalten von Modellen, zur Markergenerierung und zum Tracking vorgesehen. Eine Einschätzung der Usability des User Interfaces wird im nächsten Abschnitt vorgenommen. \\
Ein Problem das der Prototyp noch aufweist ist allerdings, dass die Kameraansicht, sowie die Modelle um 90 Grad gekippt sind.

\section{Analysierende Evaluation}
In einem zweiten Schritt sollten mit Hilfe eines Usability-Tests, bei dem potentielle Nutzer Anwendungsfälle, die sie mit dem Prototyp durchlaufen sollen, vorgelegt bekommen, Fehler und Optimierungsmöglichkeiten gefunden werden. Außerdem sollte die Anwendung im Hinblick auf ihren praktischen Nutzen getestet werden.\\ 
Bei der Erstellung und Durchführung wurde sich an dem in \cite[Kapitel 4]{hegner:evaluation} beschriebenen Verfahren eines Usability-Tests orientiert.

\subsection{Untersuchungsziel}
Das Ziel des Usability-Testes ist es den entwickelten Prototypen dieser Arbeit auf seine Nutzbarkeit zu testen. Dadurch fällt der durchgeführte Test nach \cite[S. 37]{hegner:evaluation} in die Kategorie der bewertenden Tests, die ein neues Produkt vor der weiteren Einführung bewerten sollen.


\subsection{Auswahl der Testnutzer}
Nach \cite[S. 38]{hegner:evaluation} sollte die Stichprobe eine Gruppengröße von 6 bis maximal 20 repräsentativen Nutzern umfassen. \\
Vor der Auswahl der Versuchspersonen, ist es notwendig ein Benutzerprofil zu erstellen, das die Zielgruppe der Anwendung beschreibt. \\
Für den entwickelten Prototypen ist die Hauptzielgruppe, der Bildungsbereich und dabei speziell das höhere Schulniveau und die Universität. Fachlich gesehen eigenen sich vor allem die Naturwissenschaften, mit der Medizin, der Biologie und der Chemie, zum Einsatz der Anwendung. Grundvoraussetzung ist hier das dreidimensionale Modelle Bestandteil des Lerninhaltes sind. Sekundär wäre auch ein Nutzen im Bereich der Geologie oder der Geschichte denkbar. Zu dem richtet sich diser erste Prototyp vor allem an das selbstorganisierte Lernen und somit die älteren Altersstufen, da die Modelle selbst eingefügt werden müssen und nicht direkt von einem Lehrenden innerhalb der Anwendung bereitgestellt werden können. \\
Problematisch bei der Auswahl der Versuchspersonen hat sich die Corona-Pandemie herausgestellt, da die Tests in Präsenz stattfinden mussten.
Dadurch ist zum einen die Größe und Repräsentanz der Stichprobe gesunken.\\
\todo{Beschreibung der gewählten Versuchspersonen}

\subsection{Testaufgaben}
Bei der Wahl der Aufgabe wurde sich an dem Anwendungsfall orientiert das ein Lernender ein Modell an ein Dokument heften möchte. Dazu wurden den Versuchspersonen, die folgenden problemorientierte Aufgabe gegeben:
\glqq Sie haben in Google Drive im Ordner Modelle verschiedenen dreidimensionale Modelle abgelegt. Bitte laden Sie eins dieser Modelle in der Anwendung hoch und speichern Sie den erzeugten Marker in dem Ordner Marker in Google Drive, damit Sie ihn später in ein Dokument einfügen können. Anschließend lassen Sie sich bitte ihr gewähltes Modell auf dem beiliegenden Testdokument, in dem ihr Marker bereits eingefügt worden ist anzeigen. Nun haben Sie die Möglichkeit mit der Darstellung zu experimentieren. \grqq
Damit die Testperson den Marker nicht eigenständig mit Hilfe eines externen Programms in ein Dokument einfügen muss, wird der Person ein solches Dokument bereits beigelegt. Da die Anwendung immer die nächst freie ID verwendet, kann auch bereits der Marker in dieses eingefügt werden.

\subsection{Testumgebung}
Als Testumgebung wurde nach dem Prinzip der Felduntersuchung der Arbeitsplatz also der Schreibtisch der jeweiligen Versuchspersonen gewählt, um die Anwendung in einer realistischen Arbeitsumgebung testen zu können \citep[S. 44]{hegner:evaluation}.

\subsection{Durchführung}
Die Testdurchführung hatte den folgenden Aufbau (nach \cite[S. 46]{hegner:evaluation}):
\begin{enumerate}
\item Begrüßung der Versuchspersonen.
\item Erklärung der Intention des geplanten Tests.
\item Vorlegen der Einverständniserklärung
\item Durchführen des 1. Teils des Interviews
\item Vorstellung des Prototypen und ein grober Überblick über die Funktionen und den Zweck des Prototypen.
\item Erklärung der Aufgabe.
\item Klärung von Fragen.
\item Beginn des Usability-Tests.
\item Durchführen des 2. Teils des Interviews
\item Für Teilnahme bedanken.
\end{enumerate} 

\subsection{Datenerfassung und Dokumentation}
Die Hauptmethode der Datenerfassung stellt ein Interview dar, welches in 2. Teile gegliedert ist. Der erste Teil sollte vor dem Test durchgeführt werden, um das Vorwissen und die Erwartungen zu klären. Der zweite Teil bezog sich dann auf den Usability-Test und wurde dem entsprechend im Anschluss durchgeführt. Inhaltlich sollte im zweiten Teil Probleme und Verbesserungen im Umgang mit dem Prototypen und einem Einsatz von Augmented Reality aufgedeckt werden.\\
Des weiteren wurden während der Benutzer die Aufgaben mit dem Prototypen bewältigt hat, die Beobachtungen notiert, um auftretende Probleme und Abweichungen zum erwarteten Verhalten zu dokumentieren.
\todo{Verweis auf Interviewbogen}

\subsection{Ergebnisse}
Auf Grund der Corona-Pandemie konnte der Utility-Test nur mit einer repräsentativen Versuchsperson durchgeführt werden. \todo{ Verweis dokument} \\
Diese Testperson war aus dem Bereich der (Human-) Medizin und hatte noch keine wirklichen Erfahrungen mit der Augmented Reality im Bildungsbereich gemacht.\\
Generell konnte die Aufgabe ohne wirkliche Probleme bewältigt werden. Jedoch war die Dateistruktur der Modelldateien bestehend aus einer Obj- und einer Texturdatei etwas verwirrend für die Testperson. Im nachfolgenden Interview wurde angemerkt, dass an dieser Stelle eventuell etwas mehr Erklärungen oder ein Info-Button angefügt werden könnte.\\
Auf die Frage, ob der ein Einsatz in der Praxis für die Versuchsperson in Frage kommen würde, wurden die folgenden Einsatzmöglichkeiten erläutert:
\begin{itemize}
\item Die Testperson hält in ihrem Studiengang regelmäßig Vorträge, bei denen Sie sich einen Einsatz der Technologie vorstellen könnte, in dem im Rahmen der Präsentation an den Modellen bestimmte Inhalte gezeigt oder diese im Nachhinein im Rahmen eines Handouts zur Verfügung gestellt werden könnten.
\item Außerdem fasst die Versuchsperson regelmäßig die Vorlesungen zusammen und könnte sich dabei vorstellen die Modelle beziehungsweise Marker in die Zusammenfassungen einzubauen.
\item Eine weitere Möglichkeit, die sich mehr auf den Professor bezieht, wäre es laut der Person, dass der dieser die Modelle in seine Vorlesung einbauen könnte.
\end{itemize}
Besonders hervorgehoben wurde dabei von der Versuchsperson, dass sich die zweite Einsatzmöglichkeit dann eignen würde, wenn die Modelle durch den Professor zur Verfügung gestellt werden würden.\\
Als Verbesserungen und Erweiterungen würde sich die Person noch eine Möglichkeit wünschen, das Modell zu vergrößern. Dieses deckt sich mit dem in der Vision definierten Interaktionsmöglichkeit, die im ersten Prototyp noch nicht umgesetzt wurde (vergleiche Kapitel \ref{sec:vision}).  \\
Des weiteren wurden natürlich auch die in der \nameref{sec:bewertende-evaluation} angesprochen.
Der letzte Punkt der angemerkt wurde war das auch detailreichere Modelle interessant wären, diese Anforderung bezieht sich jedoch nicht auf den Prototypen, sondern viel mehr auf die Modellauswahl.

