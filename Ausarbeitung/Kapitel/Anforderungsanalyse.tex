\chapter{Anforderungsanalyse}\label{chapter:anforderungsanalyse}
In diesem Kapitel werden die Anforderungen an die Anwendung und den zu entwickelnden Prototypen erhoben. \\
Dazu wird im folgenden wird eine Anforderungsanalyse für den zu entwickelnden Prototyp auf Grundlage der Ergebnisse der Literaturrecherche durchgeführt.
Dazu wurde die in der Softwaretechnik I  vorgestellte Struktur der Dokumentation genutzt \citep[Folie 209-214]{winter:srs-anforderungen}. 
Dieses Verfahren sieht den folgenden Aufbau vor:
\begin{enumerate}
\item Vision 
\item Machbarkeitsstudie
\item Beschreibung der Systemumgebung
\item Anwendungsfälle
\item Anforderungsliste
\item Prototypen
\item Glossar
\end{enumerate}
Im folgenden wird jedoch nur eine vereinfachte, an diese Arbeit angepasste Struktur genutzt und die Anforderungen werden direkt für den Prototypen erhoben.


\section{Vision}\label{sec:vision}
Die Anwendung hat den Bildungsbereich und das Lernen mit Hilfe von Augmented Reality als Zielgruppe. In der Arbeit von \citeauthor{diegmann:benefits-ar} werden dazu verschiedene Anwendungsarten definiert (vergleiche \ref{sec:diegmann-benefits-ar}). Diese stellt zudem auch das entdeckungsbasierte Lernen als die Einsatzmöglichkeit mit den meisten Vorteilen für das AR-gestützte Lernen heraus. In diese Kategorie fällt auch die in Abschnitt \ref{sec:atlas-humananatomie} vorgestellte Anwendung \glqq Atlas der Humananatomie 2020\grqq. \\
Die Anwendungen soll in der Art der Wissensvermittlung auf den selben, bereits im praktischen Einsatz getesteten Grundlagen beruhen. Sie soll Lernende unterstützen, indem sie ihnen dreidimensionale Modelle anzeigt. \\
Jedoch soll das Konzept von Anwendungen, wie \glqq Atlas der Humananatomie 2020\grqq{} dabei wie folgt erweitert werden:\\
Zum einem soll die Anwendung so aufgebaut werden, dass sie vom Nutzer speziell an die eigenen Bedürfnisse angepasst werden kann. Dazu soll die Anwendung kein festes Set an Modellen zur Verfügung stellen, sondern die Modelle können vom Nutzer hochgeladen werden. \\
Der zweite Punkt ist, dass anstelle des meistens verwendeten Natural Feature Tracking ein Markerbasiertes Tracking verwendet werden soll. So das die Modelle speziellen Dokumenten, Textpassagen oder Lerninhalten mit Hilfe der Marker zugeordnet werden können. Diese Marker könnten dann schnell und einfach mit der Kamera getrackt werden.\\
Außerdem ist es die Vision, dass dieses Markerbasierte Tracking in einem großen Kontext genutzt werden kann. \\
Dazu wäre eine Umsetzung angelehnt an die Webseite Socrative.com denkbar. Diese Webseite bietet Professoren die Möglichkeit Umfragen über einen Raumcode mit den Studierenden zu teilen. \\
Ein ähnliches System könnte auch zum Teilen der Modelle genutzt werden, sodass der Professor die Marker verknüpft mit den entsprechenden Modellen in eine Präsentation zusammen mit einem Raumcode einfügen kann und der Studierende dann, sobald er den Raumcode eingegeben hat, die Modelle in seinem Kamerabild sieht, wenn der entsprechende Marker getrackt wurde.\\
Ein solcher Einsatz von Augmented Reality bietet die Möglichkeit eindimensionale Vorlesungen oder Unterrichtseinheiten um eine interaktive Komponente zu erweitern und sollte viele der von \citeauthor{diegmann:benefits-ar} erwähnten positiven Auswirkungen, wie 
eine Verbesserung der  Motivation, der Aufmerksamkeit des Verständnisses der Lerninhalte übertragen können (vergleiche Kapitel \ref{sec:diegmann-benefits-ar}). \\
Ein wichtiger Faktor von Augmented Reality im Bildungsbereich ist nach \citeauthor{billinghurst:ar-in-education} und\citeauthor{diegmann:benefits-ar} auch die Interaktivität mit den Lerninhalten, um einen optimalen Lernerfolg zu erzielen (vergleiche Kapitel \ref{sec:billinghurst-ar-education} und \ref{sec:diegmann-benefits-ar}). Dazu sollte die Anwendung dem Lernenden die Möglichkeit bieten über eine Interaktion mit dem Marker mit dem Modell zu agieren. Zusätzlich sollte es auch möglich sein mit Hilfe von Gesten auf dem Smartphone das Modell zu verändern. \\
Des weiteren würden auch Faktoren, wie die von \citeauthor{billinghurst:ar-in-education} beschriebene Verbesserung der Kommunikation während der Interaktion mit digitalen Lernmethoden (vergleiche \ref{sec:billinghurst-ar-education}), unterstützt werden. 

\subsection{Der Prototyp}
Im Rahmen der der Arbeit wurde dabei das folgende Konzept für einen vertikalen Prototypen entwickelt. \\
Dieser soll die Funktionalität des Verarbeitens eigener Modelle sowie das Tracken von selbst erstellten Markern testen.
Dazu soll der Fokus auf die Umsetzung der Augmented Reality und das Erstellen der zu trackenden Marker gelegt werden. Die hochgeladenen Modelle sollen lediglich lokal gespeichert werden. \\
Dadurch handelt es sich bei dem Prototyp sozusagen um die Umsetzung eines \glqq privaten Raumes\grqq . Er bietet dem Anwender nicht die Möglichkeit die Modelle zuteilen.\\ 
Eine ansonsten notwendige Datenspeicherung auf einem Server, das Implementieren eines Raumsystems mit Zugangscode und weitere Funktionen zum Teilen von Daten fallen dadurch weg. Außerdem sollen als Endgeräte lediglich Androidgeräte betrachtet werden

\section{Beschreibung der Systemumgebung}
Bei dem Prototyp handelt es sich um eine mobile Anwendung. Das Endgerät für den Nutzer ist dementsprechend ein Smartphone oder Tablet. \\
Dabei ist das Betriebssystem auf dem die Anwendung laufen soll Android.
Der relevante Sensor, den das Endgerät für das Trackingverfahren bereitstellt, ist die Kamera. Die Anwendung muss die Videoframes analysieren und eine Ausgabe in Form von einem gerenderten 3D-Modell auf dem Display darstellen. \\
Da der Prototyp auf einem mobilen Endgerät genutzt wird, variiert die äußere Umgebung in welcher dieser zum Einsatz kommt stark. Deshalb ist es notwendig, dass das System robust gegenüber verschiedenen Umwelteinflüssen ist.\\
Eingaben durch Nutzer erfolgen über den Touchscreen und Bewegungen des Endgerätes. \\
Des weiteren benötigt der Prototyp sowohl lesenden als auch schreibenden Zugriff auf den lokalen Speicher des Endgeräts, damit der Nutzer seine Modelle hochladen und die Marker speichern kann. 


\section{Anwendungsfälle des Prototyps}
Ein Anwendungsszenario des Prototypen ist die mit dem Prototyp generierten Marker auf einem auf einem begleitenden Unterrichtsdokument einzufügen und ein oder mehrere relevante Modelle zu verlinken. Dazu werden die Modelle in dem Prototyp hochgeladen und lokal auf dem Gerät gespeichert. Die von der Anwendung generierten Marker können dann in einem Dokument eingefügt werden. Über die Kamerafunktion ist es im Anschluss möglich die Marker zu scannen, um sich die entsprechenden, dreidimensionale Modell anzeigen zu lassen.\\
Insgesamt besitzt der Prototyp drei Unterfunktionen: 
\begin{enumerate}
\item Das Generieren von Markern. Hierbei erstellt der Prototyp Marker für den Anwender, die alle eine unterschiedliche ID besitzen und aus jeder Rotation eindeutig zuerkennen sind. Dafür müssen sich zwei Marker auch unterscheiden, wenn einer von ihnen beliebig rotiert wurde.
\item Das Tracken von Markern. Bei dem die Anwendung einen Marker in einem Videoframe erkennt und seine Position und Transformation im Videoframe berechnet. Dabei liegt der Marker in gedruckter Form auf einem Blatt Papier vor oder wird auf einem Bildschirm angezeigt.
\item Das Rendern von Modellen. Das anzuzeigende Modell wurde dabei vom Nutzer  als Datei hochgeladen und muss zunächst vom System verarbeitet werden. Um im Anschluss das Modell realistisch in Relation zum Marker zu platzieren, muss die beim Tracking berechnete Position und Transformation des Markers verwendet werden. 
\end{enumerate} 
Allgemein lassen sich die Funktionen und Anwendungsfälle in dem in Abbildung \ref{fig:Use-Cases} gezeigtem Anwendungsfalldiagramm visualisieren.
\begin{figure}[h!]
\centering
\includegraphics[width=1.0\textwidth]{Abbildungen/use-case-diagram.png}
\caption[Use Cases des Prototyps]{Use Case Diagramm des Prototyps. (Quelle: Eigene Darstellung)}
\label{fig:Use-Cases}
\end{figure}


\section{Anforderungsliste}
Im folgenden Abschnitt werden die Anforderungen an den Prototyp definiert. Dabei wird zwischen funktionalen und nicht-funktionalen Anforderungen unterschieden. 
\begin{quote}
\glqq Eine funktionale Anforderung ist eine Anforderung bezüglich des Ergebnisses eines Verhaltens, das von einer Funktion des Systems [...] bereitgestellt werden soll.\grqq \citep[S. 17]{rupp:requirements}
\end{quote}
Nicht-funktionale Anforderungen umfassen im Vergleich dazu die Anforderungen an die Technologie, die Qualität, die Benutzeroberfläche, die durchzuführende Tätigkeiten und rechtlich vertragliche Anforderungen \citep[S. 17]{rupp:requirements}.\\
Sie können als Eigenschaften, die das System besitzen muss, um den Nutzer zufrieden zustellen, beschrieben werden. \citep[S. 10]{robertson:requirements-process}. Diese Eigenschaften beziehen sich meistens auf die Bereiche Service Level, Zugriffsbeschränkungen, Sicherheit, Monitoring, Kontrolle, Schnittstellen, Archivierung, Benutzerfreundlichkeit und Konversion \citep[S. 139]{boehm:systementwicklung}. \\
Auf Grundlage dieser Definition wurden die folgenden funktionalen und nicht-funktionalen Anforderungen mit Hilfe der von \citeauthor[S. 219]{rupp:requirements} definierten Anforderungsschablonen aufgestellt:

\subsection{Funktionale Anforderungen}
\begin{itemize}
\item[FA01] Die Anwendung muss fähig sein, einen Marker zu generieren.
\item[FA02] Die Anwendung muss fähig sein, Marker in dem Kamerabild erkennen zu können.
\item[FA03] Die Anwendung muss fähig sein, generierte Marker zu unterscheiden.
\item[FA03] Die Anwendung muss dem Benutzer die Möglichkeit bieten, den generierten Marker auf dem lokalen Gerätespeicher zu speichern.
\item[FA04] Die Anwendung muss dem Benutzer die Möglichkeit bieten, 3D-Modelle in Form von OBJ-Dateien aus dem lokalen Speicher hochladen zu können.
\item[FA05] Die Anwendung muss dem Benutzer die Möglichkeit bieten, Textur-Dateien im Bildformat(jpeg, png) aus dem lokalen Speicher hochladen zu können.
\item[FA06] Die Anwendung muss fähig sein, hochgeladene Dateien in der Anwendung zu speichern.
\item[FA07] Die Anwendung muss fähig sein, jedem Modell einen individuellen Marker zuzuordnen.
\item[FA08] Die Anwendung muss fähig sein, die dreidimensionalen Modelle im Kamerabild anzuzeigen.
\item[FA09] Die Anwendung muss fähig sein, die Transformation der Marker zu berechnen.
\item[FA10] Die Anwendung muss fähig sein, die Modelle basierend auf der berechneten Transformation zu rendern.
\end{itemize}

\subsection{Nichtfunktionale Anforderungen}
\begin{itemize}
\item[NF01] Das Endgerät der Anwendung sollte ein Android Smartphone(Huawei P30 Pro) sein.
\begin{enumerate}
\item[NF01.1] Die genutzte Entwicklungsumgebung der Anwendung sollte Android Studio sein.
\item[NF01.2] Die genutzte Programmiersprache der Anwendung sollte Java sein.
\end{enumerate}
\item[NF02] Die Markergenerierung der Anwendung sollte so gestaltet sein, dass es mindestens 10 verschiedene Marker generieren kann.
\item[NF03] Das Tracking der Anwendung sollte kamerabasiert sein.
\item[NF04] Das Tracking der Anwendung sollte auf einem markerbasiertem Verfahren beruhen.
\item[NF05] Das Tracking der Anwendung sollte mit mindestens 30FPS laufen.
\item[NF06] Das Tracking der Anwendung sollte so gestaltet sein, dass es alle generierten Marker der Anwendung erkennen kann.
\item[NF07] Bei vollständig erkanntem Marker sollte das Tracking robust gegenüber äußerlichen Veränderungen sein.
\begin{enumerate}
\item[NF07.1] Bei vollständig erkanntem Marker sollte das Tracking robust gegenüber Rotation sein.
\item[NF07.2] Bei vollständig erkanntem Marker sollte das Tracking robust gegenüber Skalierung
sein.
\item[NF07.3] Bei vollständig erkanntem Marker sollte das Tracking robust gegenüber Perspektive sein.
\item[NF07.4] Bei vollständig erkanntem Marker sollte das Tracking robust gegenüber Belichtung sein.
\end{enumerate}

\item[NF08] Die Benutzeroberfläche der Anwendung sollte eine einfache Navigation zu allen Komponenten der Anwendung bereitstellen.
\item[NF09] Die Benutzeroberfläche der Anwendung sollte eine Ansicht zum Hochladen und Verwalten der Modelle bieten.
\item[NF10] Die Benutzeroberfläche der Anwendung sollte einen Zugriff auf die Kamera ermöglichen.








\end{itemize}

