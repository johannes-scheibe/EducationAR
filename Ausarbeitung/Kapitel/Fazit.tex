\chapter{Fazit}\label{chapter:fazit}
Dieses Kapitel zieht ein Fazit über die Ergebnisse und den Verlauf dieser Arbeit. Zudem wird ein Ausblick gegeben.

\section{Zusammenfassung}\label{sec:zusammenfassung}
Augmented Reality stellt eine spannende Möglichkeit zur Verbesserung des Unterrichts dar. Richtig eingesetzt, bietet es die Möglichkeit, nicht nur die Motiavtion, die Ergebnisse und Fähigkeiten der Lernenden zu Verbessern, sondern kann auch die Lehrstrukturen verbessern, in dem es erlaubt, die starren Strukturen eindimensionaler Unterrichtmethoden zu durchbrechen. Es ermöglicht einseitigen Vorträgen der Lehrenden eine interaktive Komponente zu verleihen, durch welche die Lernenden Wissenskonzepte selbst erforschen können.\\
Dabei werden momentan viele spannende Konzepte und Ideen zum Einsatz von Augmented Reality in der Bildung erforscht, deren genauer Mehrwert vermutlich letztendlich weitere Studien und Praxistests benötigt. \\
Im Rahmen dieser Arbeit wurde dann auf Grundlage existierender Arbeiten eine eigenen Augmented Reality Anwendung für den Bildungsbereich entwicklt. In der Art der Wissensvermittlung orientiert sich die Anwendung dabei an Beispielen aus der Praxis wie \glqq Atlas der Humananatomie\grqq . \\
Dabei grenzt sich der entwickelte Prototyp aber durch den Einsatz von Markern und dem Fokus auf eigene Modelle von existierenden Lösungen ab. Durch den Einsatz dieser Marker ist es möglich die Modelle mit anderen Formen der Wissensdarstellung zu verknüpfen. So kann der Marker beispielsweise an Textpassagen angehängt werden, die dem Nutzer weitere Informationen liefern. Dieses ist vor allem dann relevant, wenn man versucht verschiedene Lehrtechniken zu kombinieren. \\
Das Hochladen eigener Modelle ermöglicht zudem eine deutlich bessere Anpassbarkeit an die Anforderungen der Vorlesung und den Endnutzer. Jedoch geht dieses mit der Herausforderung der Beschaffung beziehungsweise Bereitstellung der Modelle einher.\\
Die Evaluation hat dabei gezeigt, dass zum einen die definierten Anforderungen erfüllt werden können und zum anderen konnte mit dem Usability-Test gezeigt werden, dass der Einsatz einer solchen Anwendung in der Praxis durchaus realistisch ist. 

\section{Kritische Reflexion}\label{sec:reflexion}
Insgesamt sind die Ergebnisse der Arbeit als durchaus positiv zu bewerten und auch das Endprodukt, der entwickelten Prototyp, konnte in großen Teilen die Vorstellungen im Vorfeld der Arbeit erfüllen. Ärgerlich ist dabei jedoch, dass die Probleme bezüglich des Portrait-Modus nicht behoben werden konnten. Diese Probleme wurden in der ARToolKitX Beispielanwendung mit Hilfe einer Quick\& Dirty-Lösung behoben, die jedoch aufgrund der Neuerungen, wie der Navigation, im Prototypen nicht mehr genutzt werden konnte. Vor allem an dieser Stelle hat sich die mangelnde Dokumentation von ARToolKitX bemerkbar gemacht. Diese Tatsache zusammen mit dem Problem, das ebenfalls kaum Forumeinträge zu ARToolKit spezifischen Problemen existieren, hat extrem viel Zeit in Anspruch genommen. \\
So bestand ein Großteil allein darin die Strukturen von ARToolKitX zu verstehen, indem der Programmcode der Klassen durchgearbeitet wurde.\\
Ein weiterer Punkt der leider nicht wie gewünscht umgesetzt werden konnte bezog sich auf die Evaluation. Hier mussten die Tests mit dem Nutzer aufgrund der Kontaktbeschränkungen in einem deutlich geringeren Rahmen durchgeführt werden. Geplant war es an dieser Stelle eigentlich mit verschiedenen Nutzergruppen, wie Professoren, Tutoren und Studenten aus verschiedenen Fachbereichen zu testen. Dieses konnte leider nicht umgesetzt werden.\\
Betrachtet man Rückblickend den gesamten Verlauf der Arbeit, ist der Start durch aus kritisch zu betrachten. Dort fehlte es an einem konkreten Startdatum, wodurch es nie wirklich den Punkt gab, ab dem mit Blick auf ein konkretes Ende hingearbeitet wurde. Ein Punkt der vermutlich dazu beigetragen hat, ist das die Arbeit zu Beginn der Corona-Pandemie begonnen wurde und die ganzen internen Abläufe der Universität Oldenburg überarbeitet werden mussten.\\

\section{Ausblick}\label{sec:ausblick}
Der entwickelte Prototyp stellt nur eine erste Version der in der \nameref{chapter:anforderungsanalyse} definierten Vision dar, welche noch in vielen Bereichen erweitert werden kann. \\
Ein Punkt der auch in dem Usability-Test wiederfindet ist die Erweiterung um eine Interaktionsmöglichkeit. Hier könnte mit Hilfe von Gestend das Modell vergrößert oder gedreht werden. \\
Der nächste große Schritt wäre es dann den Nutzern die Möglichkeit zugeben die Modelle mit anderen Nutzern zu teilen. Dazu könnte Beispielsweise das in der \nameref{sec:vision} definierte Konzept mit einem Raumcode umgesetzt werden. Diese Eigenschaft ist auch essentiell, um das Anwendungsszenario, bei dem der Professor die Modelle und Marker in seine Vorlesung einbaut umzusetzen. Dieses wurde sowohl zu Beginn bei der \nameref{chapter:anforderungsanalyse} definiert als auch bei der Nutzer-Evaluation herausgearbeitet. \\
Neben den konkreten Umsetzungen ist auch ein weiteres Erforschen der speziellen Anforderungen des Bildungsbereiches sinnvoll. Der Prototyp stellt eine erste Grundlage dar, für den in einem nächsten Schritt die Anforderungen und der konkrete Anwendungsfall verfeinert werden sollte. Dazu wären unter anderem eine konkrete Anforderungserhebung in der Praxis und mit Lehr- und Fachkräften denkbar. 
