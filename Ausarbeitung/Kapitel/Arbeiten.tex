\chapter{Verwandte Arbeiten}\label{chapter:arbeiten}

\section{A Systematic Review of Learning through Mobile Augmented Reality}
In der wissenschaftlichen Arbeit von \citeauthor{hedberg:review-ar-learning} \citep{hedberg:review-ar-learning} wurde mit Hilfe einer systematischen Literaturauswertung eine Einschätzung des Einsatzes von Augmented Reality zu Bildungszwecken vorgenommen. \\
Dabei wurde die Relevanz einzelner Einsatzgebiete anhand der Anzahl an Veröffentlichungen zum jeweiligen Thema gemessen.\\ Die Veröffenntlichungen stammen dabei aus den Datenbanken \glqq IEEE Xplore\grqq , \glqq Elsevier\grqq und der \glqq ACM digital libary\grqq und wurden über Schlüsselwörter herausgesucht. Die Ergebnisse der Suche wurden im Anschluss noch gefiltert und kategorisiert, um irrelevante Veröffentlichungen auszuschließen.
\subsection{Ergebnisse}
Anhand der gesammelten wissenschaftlichen Arbeiten kam die Studie zu den folgenden Ergebnissen in den für diese Arbeit relevanten Kategorien:

\subsubsection{Bildungsniveau}
Diese Kategorie zielte darauf ab die Zielgruppen der AR-Systeme herauszufinden. Dabei kam die Studie zu dem Ergebnis, dass die meisten Systeme (fast 50 \%) bezogen auf das amerikanische Bildungssystem für die Universität geschaffen wurden. Mit einem deutlichen Abstand folgt die Grundschule, die Junior High School, die High School und die Vorschule.

\subsubsection{Mobiles Endgerät}
In dieser Kategorie ging es darum herauszufinden, welches die meist genutzten Endgeräte für Augmented Reality im Bildungsbereich darstellen. Hierbei kam heraus, dass das Handy bzw. das Smartphone (43,84 \%) gefolgt von dem Tablet (27,40 \%) diese Kategorie dominieren, wobei ebenfalls 27 \% der wissenschaftlichen Arbeiten keine spezifische Plattform angaben.

\subsubsection{Unterrichtsfach}
Diese Kategorie untersuchte die fachliche Ausrichtung der Augmented Reality und kam zu dem Ergebnis das die Systeme vor allem in den Naturwissenschaften genutzt werden. Darauf folgen mit signifikantem Abstand die Sprachen, Geschichte und Technik.

\subsubsection{Lernerfolge}
Hier bei wurden Studien untersucht, die sich mit dem pädagogischen Nutzen von Augmented Reality beschäftigt haben. Dabei kamen von 73 Studien lediglich zwei nicht zu dem Ergebnis, dass AR einen positiven Einfluss auf das Lernen besitzt.\\
Allgemein stellten 45 Veröffentlichungen (54,88 \%) eine erhöhte Motivation und ein verbessertes Engagement fest. 28 Studien (34,15 \%)konnten verbesserte Lernergebnisse bei den Studenten aufzeigen.  

\subsubsection{Pädagogische Methoden}
In der letzten Kategorie, wurden die Einsatzarten von Augmented Reality im Bildungsbereich untersucht und das Ergebnis erarbeitet, dass die drei Hauptanwendungsmethoden das interaktive, das forschungsbasierte und das kollaborative Lernen sind.

\section{Augmented Reality in Education}
\citeauthor{billinghurst:ar-in-education} untersuchte in seiner wissenschaftlichen Arbeit \citep{billinghurst:ar-in-education} den Einsatz von Augmented Reality im Bildungsbereich.\\
Dabei geht er in den folgenden drei Unterkapiteln auf unterschiedliche Eigenschaften der AR im Bildungsbereich ein.
\subsection{Nahtlose Interaktion}
\citeauthor{billinghurst:ar-in-education} führt hier auf, dass Schüler besser zusammenarbeiten, wenn sie sich gemeinsam auf einen Arbeitsplatz fokussieren. Dieses ist bei computerbasierten Lernen schwierig um zusetzten, da sich jeder auf den Bildschirm vor sich fokussiert. Dadurch fehlt eine wichtige Eigenschaft, die die Kommunikation in der Gruppe verbessert: die gegenseitige Sichtbarkeit. Wenn Schüler gleichzeitig das Objekt der Diskussion und ihre Diskussionspartnern sehen, werden auch die nichtverbalen Gesprächsmerkmale, wie Gesten oder die Mimik wahrgenommen. Diese Merkmale bilden einen essentiellen Teil der menschlichen Kommunikation. \\
Durch den Einsatz von Augmented Reality können diese Eigenschaften bei behalten werden und trotzdem gleichzeitig computerbasierte Inhalte angezeigt werden.
\subsection{Greifbare Schnittstelle}
Hier führt \citeauthor{billinghurst:ar-in-education} auf, dass im Bildungsbereich physische Objekte dazu genutzt werden Bedeutung von theoretischem Wissen zu übermitteln, in dem sie die Eindrücke des Schülers, durch ihre Erscheinung, ihre physkalischen Eigenschaften, ihren räumlichen Beziehungen und der Fähigkeit, die Aufmerksamkeit zu fokussieren, verstärken.\\
Diese Eigenschaften lassen sich zu großen Teilen auch auf die virtuellen Objekte in der Augmented Reality beziehen. Auf Grund der direkten Beziehungen zwischen der virtuellen Objekten und der Augmented Reality ist zum Beispiel eine physikalisch basierte Interaktion mit den computergenerierten Objekten möglich.
\subsection{Übergangsschnittstelle}
Im dritten Abschnitt bezieht sich \citeauthor{billinghurst:ar-in-education} auf das bereits in Kapitel \ref{sec:ar} eingeführte RV-Kontinuum und führt an, dass man mittels AR den Nutzer entlang des Kontinuums in die virtuelle Welt führen kann. Besonders Kinder können so in die Seiten eines Buches eintauchen und die Fantasie Realität werden lassen. Dadurch werden aus statischen Unterrichtsbüchern dynamische interaktive Umgebungen.

