\chapter{Einleitung}\label{chapter:einleitung}
Das folgende Kapitel bildet die Grundlage dieser Arbeit. Es beleuchtet die Thematik und die daraus resultierende Forschungsfrage. Des weiteren werden die Ziele, sowie das Vorgehen der Arbeit definiert. Abschließend folgt ein Überblick über den Aufbau der Arbeit.

\section{Motivation}\label{sec:motivation}
In der heutigen Zeit werden immer neuere Methoden zur Informationsbeschaffung und -darstellung entwickelt, bei denen sich oft die Frage stellt, welche praktische Relevanz diese Verfahren besitzen. \\
Augmented Reality stellt eine dieser Technologien dar, die momentan immer mehr an Relevanz gewinnt. Sie bildet den Oberbegriff für die Erweiterung der Realität mit virtuellen Objekten (vgl. Kapitel \ref{sec:ar}). \\
Eines der wohl verbreitetsten Anwendungsgebiete stellt dabei der Entertainmentbereich dar. Hier wird die Technologie vor allem im Bereich der Spielentwicklung für Mobilgeräte genutzt. Die grundlegenden Technologien und Methoden lassen sich jedoch auch auf andere Bereiche übertragen. \\
Im (Hoch-)Schulumfeld könnte Augmented Reality eine neue Möglichkeit zur Visualisierung von Lerninhalten bieten, deren Vorteil darin liegt, dass es zweidimensionale Abbildungen durch dreidimensionale Modelle ersetzen zu könnte. \\
Diese Arbeit beschäftigt sich dabei mit der Thematik, wie sich die Technologien aus den bereits sehr verbreiteten Einsatzgebieten für den Bildungsbereich nutzen lassen können. \\
Auf dieser Grundlage wurde die folgende, allgemeine Forschungsfrage aufgestellt:\\
\glqq Wie lassen sich die Konzepte der Augmented Reality für den Bildungsbereich nutzen?\grqq

\section{Zielsetzung der Arbeit}\label{sec:ziele}
Das Ziel dieser Arbeit ist die Erhebung von verschiedenen Anwendungsmöglichkeiten der Augmented Reality Konzepte für den Bildungsbereich. Dazu müssen bestehende Technologien und Methoden erfasst werden, aus denen dann gegebenenfalls im Anschluss neue Konzepte abgeleitet werden. \\
Auf dieser Grundlage soll dann eine solche Anwendung entworfen und die Realisierbarkeit durch eine prototypische Umsetzung validiert werden.

\section{Vorgehensweise}

\section{Aufbau der Arbeit}
