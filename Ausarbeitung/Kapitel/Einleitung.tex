\chapter{Einleitung}\label{chapter:einleitung}
Diese Arbeit beschäftigt sich mit der Entwicklung einer Augmented Reality Anwendung für den Bildungsbereich. Im folgenden wird die Thematik und die daraus resultierende Forschungsfrage genauer beleuchtet. Des weiteren werden die Ziele, sowie das Vorgehen der Arbeit definiert. Abschließend folgt ein Überblick über den Aufbau der Arbeit.

\section{Motivation}\label{sec:motivation}
In der heutigen Zeit werden immer neuere Methoden zur Informationsbeschaffung und -darstellung entwickelt, bei denen sich oft die Frage stellt, welche praktische Relevanz diese Verfahren besitzen. \\
Augmented Reality stellt eine dieser Technologien dar, die momentan immer mehr an Relevanz gewinnt. Sie bildet den Oberbegriff für die Erweiterung der Realität mit virtuellen Objekten (vgl. Kapitel \ref{sec:ar}). \\
Begünstigt wird diese Entwicklung vor allem durch den technischen Fortschritt im Bereich der Mobilgeräte. Diese ermöglichen Privatpersonen im alltäglichen Leben den Zugriff auf eine enorme Rechenleistungen und öffnen somit die Tür zu neuen Technologie, wie der Augmented Reality. \\
Eines der wohl verbreitetsten Anwendungsgebiete stellt dabei der Entertainmentbereich dar. Hier wird die Technologie vor allem im Bereich der Spielentwicklung für Mobilgeräte genutzt. Die grundlegenden Technologien und Methoden lassen sich jedoch auch auf andere Bereiche übertragen. \\
Im (Hoch-)Schulumfeld kann Augmented Reality eine neue Möglichkeit zur Visualisierung von Lerninhalten bieten, deren Vorteil unter anderem darin liegt, zweidimensionale Abbildungen durch dreidimensionale Modelle zu ersetzen. \\
So ist es mit Augmented Reality möglich den Lernenden einen weiteren Zugang zu den Lerninhalten zu bieten. \\
Solche alternativen, digitalen Lernmethoden werden in der heutigen Zeit immer wichtiger. Auch das 2019 ausgebrochene Covid-19 führt dazu, dass die Lehre zunehmend ins Digitale verschoben und neue Lernmethoden, speziell für die Online-Lehre, gefunden werden müssen. \\
Vor diesem Hintergrund wurde die folgende Forschungsfrage aufgestellt:\\
\glqq Wie lassen sich die Konzepte der Augmented Reality für den Bildungsbereich nutzen?\grqq \\
Dabei beschäftigt sich diese Arbeit sowohl mit den konkreten Anwendungsfällen von Augmented Reality, als auch mit der Thematik, wie sich die Technologien und Verfahren der Augmented Reality für den Bildungsbereich nutzen lassen. \\



\section{Zielsetzung}\label{sec:ziele}
Das Ziel dieser Arbeit ist die Entwicklung einer Augmented Reality Anwendung für den Bildungsbereich. \\
Im Rahmen dieser Zielsetzung sollen bestehende Technologien, Methoden und Ideen zur Umsetzung von Augmented Reality im Bildungsbereich erfasst und evaluiert werden. \\
Auf dieser Grundlage soll dann eine eigenes Konzept für eine AR-Anwendung entworfen und die Realisierbarkeit durch eine prototypische Umsetzung validiert werden. \\
Der Prototyp soll dabei auf die grundlegenden Funktionen der Augmented Realitiy aufbauen und im Anschluss bezüglich der Einsetzbarkeit in der Praxis evaluiert werden. 

\section{Methodik}
Diese Arbeit verfolgt die typischen Phasen der Softwareentwicklung, bestehend aus Anforderungsentwicklung, Entwurf, Implementierung und Evaluation.\\
Dazu wurde in einem ersten Schritt eine qualitative Untersuchung der Einsatzmöglichkeiten durchgeführt. 
Diese beruhte auf einer Literaturrecherche, die sich auf die bestehenden Einsatzmöglichkeiten, sowie der Vorteile für den Bildungsbereich, konzentrierte. Bei der betrachteten Literatur beruhte zum einem auf dem Sortiment des \glqq Oldenburgerischen Regionalen Bibliotheks- und Informationssystems (ORBISplus)\grqq{} und zum Anderen auf den Ergebnisse einer Stichwortsuche im wissenschaftlichen Suchportal Google Scholar. 
Insgesamt wurden sowohl wissenschaftliche Bücher, als auch Studien, und konkrete Anwendungen herangezogen.\\
Auf dieser Basis wurde im Rahmen der Anforderungsanalyse eine Anforderungsfall für den Einsatz von Augmented Reality im Bildungsbereich entwickelt. Mit Hilfe der zugehörigen Use Cases und den Ergebnissen der Literaturrecherche konnten dann die Anforderungen an den Prototypen abgeleitet werden. Diese wurden in in funktionale und nicht-funktionale Anforderungen unterteilt. Zu dem wurde auf eine präzise Formulierung und eine Überprüfbarkeit der Anforderungen wert gelegt.\\
\todo{Vorgehen bei Entwurf?}
Während der Implementierung wurde auf einen zyklischer Testprozess zu Validierung und Dokumentierung der Ergebnisse zurückgegriffen. Dieser sah einen Test nach der erfolgreichen Implementierung einer neuen Funktion vor, um diese zu testen und eine Vergleichbarkeit zu älteren Versionen herzustellen. Dadurch sollten mögliche Mängel und Probleme durch neue Features frühzeitig erkannt werden. \\
Die Evaluierung des Prototyps wurde in zwei Teile unterteilt. Im ersten Schritt wurde die Anwendung anhand der gestellten Anforderungen evaluiert. In diesem Schritt ging es darum zu testen, ob die Anforderungen erfüllt wurden und ob sich das Konzept eventuell im Laufe der Entwicklung geändert hat. \\
Der zweite Schritt bestand aus einer Usability-Evaluation mit ausgewählten Nutzern. Dabei sollte zum einen die Umsetzung, sowie der Einsatz der durch den Prototyp realisiert Funktionen in der Praxis getestet werden. 
\todo{Methode und Vorgehen -Interview, Fragebogen}

\section{Aufbau der Arbeit}
Insgesamt besteht diese Arbeit aus 8 Kapiteln.\\
Zu Beginn dieser Arbeit werden in dem Kapitel \ref{chapter:grundlagen} die Grundlagen dieser Arbeit beleuchtet. Dabei geht das Kapitel auf die grundlegenden Prinzipien von OpenGl, der Augemnted Reality und der Android App Entwicklung ein.\\
Anschließend folgt das Kapitel \ref{chapter:arbeiten} in dem verwandte Arbeiten zum Thema dieser Arbeit beschrieben werden.\\
Diese beiden Kapitel bilden die Grundlage für die folgende Entwicklung des Prototyps. Im Kapitel \ref{chapter:anforderungsanalyse} werden die Anforderungen für diesen erhoben und der Anwendungsfall spezifiziert.\\
Im folgenden Kapitel \ref{chapter:entwurf} werden aus den Anforderungen erste Designentscheidungen abgeleitet und ein Entwurf für die Implementierung entwickelt.\\
Die Konkrete Implementierung wird in Kapitel \ref{chapter:implementierung} behandelt. Dabei werden die einzelnen, konkreten Klassen, deren Methoden, sowie die Beziehungen zwischen den Klassen beschrieben. \\
Im Kapitel \ref{chapter:evaluation} folgt eine Evaluation des Prototyps. Dazu werden zunächst die in Kapitel \ref{chapter:anforderungsanalyse} erhobenen Anforderungen betrachtet und die Realisierung durch den Prototypen geprüft. Im Anschluss wird die durchgeführte Usability-Evaluation und deren Ergebnisse beschrieben. \\
Das abschließende Kapitel \ref{chapter:fazit} gibt ein inhaltliche Zusammenfassung über die Ergebnisse dieser Arbeit und einen Ausblick auf weiterführende Aufgaben. Des weiteren wird in diesem Kapitel auch eine kritische Reflexion des Verlaufs und der Ergebnisse dieser Arbeit abgeben.


