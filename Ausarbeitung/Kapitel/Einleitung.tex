\chapter{Einleitung}\label{chapter:einleitung}
Diese Arbeit beschäftigt sich mit der Entwicklung einer Augmented Reality Anwendung für den Bildungsbereich. Im folgenden wird die Thematik und die daraus resultierende Forschungsfrage genauer beleuchtet. Des weiteren werden die Ziele, sowie das Vorgehen der Arbeit definiert. Abschließend folgt ein Überblick über den Aufbau der Arbeit.

\section{Motivation}\label{sec:motivation}
In der heutigen Zeit werden immer neuere Methoden zur Informationsbeschaffung und -darstellung entwickelt, bei denen sich oft die Frage stellt, welche praktische Relevanz diese Verfahren besitzen. \\
Betrachtet man die Entwicklung der Anzahl Nutzern auf mobilen Endgeräten, wird deutlich, dass auch die Augmented Reality zu den Technologien gehört, die stetig an Relevanz gewinnen \citep{statista:ar-users}. Sie bildet den Oberbegriff für die Erweiterung der Realität mit virtuellen Objekten (vgl. Kapitel \ref{sec:ar}). \\
Begünstigt wird diese Entwicklung vor allem durch den technischen Fortschritt im Bereich der Mobilgeräte. Dieser ermöglicht es Privatpersonen im alltäglichen Leben auf eine sehr hohe Rechenleistungen zuzugreifen und öffnet somit die Tür zu neuen Technologie, wie der Augmented Reality. \\
Eines der wohl verbreitetsten Anwendungsgebiete stellt dabei der Entertainmentbereich dar. Hier wird die Technologie vor allem im Bereich der Spielentwicklung für Mobilgeräte genutzt. Die grundlegenden Technologien und Methoden lassen sich jedoch auch auf andere Bereiche übertragen. \\
Im (Hoch-)Schulumfeld kann Augmented Reality eine neue Möglichkeit zur Visualisierung von Lerninhalten bieten, deren Vorteil unter anderem darin liegt, zweidimensionale Abbildungen durch dreidimensionale Modelle zu ersetzen, zu ergänzen oder zu erweitern. \\
So ist es mit Augmented Reality möglich den Lernenden einen weiteren Zugang zu den Lerninhalten zu bieten. \\
Solche alternativen, digitalen Lernmethoden werden in der heutigen Zeit immer wichtiger. Auch das 2019 ausgebrochene Covid-19 führt dazu, dass die Lehre zunehmend in das Digitale verschoben und neue Lernmethoden, speziell für die Online-Lehre, gefunden werden müssen. \\
Vor diesem Hintergrund wurde die folgende Forschungsfrage für diese Arbeit aufgestellt:\\
\glqq Wie lassen sich die Konzepte der Augmented Reality für den Bildungsbereich nutzen?\grqq \\
Dabei beschäftigt sich diese Arbeit sowohl mit den konkreten Anwendungsfällen von Augmented Reality, als auch mit der Thematik, wie sich die Technologien und Verfahren der Augmented Reality für den Bildungsbereich nutzen lassen. 

\section{Zielsetzung}\label{sec:ziele}
Das Ziel dieser Arbeit ist die Entwicklung einer Augmented Reality Anwendung für den Bildungsbereich. \\
Im Rahmen dieser Zielsetzung sollen bestehende Technologien, Methoden und Ideen zur Umsetzung von Augmented Reality im Bildungsbereich erfasst und die Vorteile für die Lehre aufgegriffen werden. \\
Dieses dient im Folgenden als Grundlage für die Konzeptionierung der Anwendung. Anhand der Erkenntnisse bestehender Studien und konkreter Anwendungsfälle soll ein Konzept zur Umsetzung einer Augmented Reality Anwenung für den Bildungsbereich entwickelt werden, dessen Realisierbarkeit durch eine prototypische Umsetzung validiert und praktischen Nutzen in einem letzten evaluiert werden soll.\\
Insgesamt werden die folgenden Ziele angestrebt:
\begin{itemize}
\item Erfassung des aktuellen Standes
\item Konzeptionierung einer Augmented Reality Anwendung
\item Implementierung eines Prototypen
\item Evaluation des Prototypen mit Blick auf den Bildungsbereich
\end{itemize}

\section{Methodik}
Diese Arbeit verfolgt die typischen Phasen der Softwareentwicklung, bestehend aus Anforderungserhebung, Entwurf, Implementierung und Evaluation.\\
Dazu wurde in einem ersten Schritt eine qualitative Untersuchung des aktuellen Standes der Augmented Reality im Bildungsbereich durchgeführt. 
Diese beruhte auf einer Literaturrecherche, die sich auf die bestehenden Einsatzmöglichkeiten, sowie der Vorteile für den Bildungsbereich, konzentrierte. Bei der betrachteten Literatur wurde zum einem auf das Sortiment des \glqq Oldenburgerischen Regionalen Bibliotheks- und Informationssystems (ORBISplus)\grqq{} und zum anderen auf die Ergebnisse einer Stichwortsuche im wissenschaftlichen Suchportal \glqq Google Scholar\grqq{} zurückgegriffen. 
Insgesamt wurden sowohl wissenschaftliche Bücher, als auch Studien, und konkrete Anwendungen herangezogen.\\
Anhand der Ergebnissen von Studien, die den Einsatz von Augmented Reality Anwendungen untersuchten, bestehenden Einsatzmöglichkeiten und den Vorteilen von AR-Anwendungen im Bildungsbereich konnte ein Anwendungsfall für eine Augmented Reality Anwendung im Rahmen der Anforderungsanalyse entwickelt werden.\\
Mit Hilfe der zugehörigen Use Cases und den Ergebnissen der Literaturrecherche konnten anschließend die Anforderungen an den Prototypen abgeleitet werden. Diese wurden in funktionale und nicht-funktionale Anforderungen unterteilt. Zu dem wurde auf eine präzise Formulierung und eine Überprüfbarkeit der Anforderungen wert gelegt.\\
Während der Implementierung wurde auf einen zyklischer Testprozess zur Validierung und Dokumentation der Ergebnisse zurückgegriffen. Dieser sah einen Test nach der erfolgreichen Implementierung einer neuen Funktion vor, um diese zu evaluieren und eine Vergleichbarkeit zu älteren Versionen herzustellen. Dadurch sollten mögliche Mängel und Probleme durch neue Features frühzeitig erkannt werden. \\
Abschließend folgte die finale Evaluierung des Prototyps, welche in zwei Teile aufgegliedert wurde. Im ersten Schritt wurde die Anwendung mit Hilfe einer bewertenden Evaluation anhand der gestellten Anforderungen geprüft. Dabei ging es darum zu testen, ob die Anforderungen erfüllt wurden und ob sich das Konzept eventuell im Laufe der Entwicklung geändert hat. \\
Der zweite Schritt bestand aus einer Usability-Evaluation mit ausgewählten Nutzern. Dabei sollte zum einen die Umsetzung, sowie der Einsatz in der Praxis analysiert werden. 
Zum anderen sollten mit Hilfe eines Nutzerinterviews, bei dem die Versuchsperson sowohl vor, als auch nach Benutzung des Prototyps befragt wurde, auch die Probleme und Optimierungspotentiale erfasst werden.

\section{Aufbau der Arbeit}
Insgesamt besteht die Arbeit aus 8 Kapiteln.\\
Zu Beginn werden die Grundlagen dieser Arbeit erklärt. Dabei geht das Kapitel auf die grundlegenden Prinzipien von OpenGl, der Augemnted Reality und der Android App Entwicklung ein (siehe Kapitel \ref{chapter:grundlagen}). \\
Anschließend folgt das Kapitel \ref{chapter:arbeiten}, welches die verwandte Arbeiten beleuchtet und in den Kontext dieser Arbeit einordnet.\\
Auf dieser Grundlage baut die folgende Entwicklung der Anwendung auf. Im Kapitel \ref{chapter:anforderungsanalyse} werden die Ziele der Anwendung entwickelt und die Anforderungen an den Prototypen erhoben.\\
Im folgenden Kapitel \ref{chapter:entwurf} werden aus den Anforderungen konkrete Designentscheidungen abgeleitet und ein Entwurf für die Implementierung entwickelt.\\
Letztere wird in dem Kapitel \ref{chapter:implementierung} behandelt. Dabei werden die einzelnen Klassen, sowie deren Methoden und Beziehungen beschrieben. \\
Abschließend folgt die eine Evaluation des Prototyps (siehe Kapitel \ref{chapter:evaluation}). Dazu werden die in Kapitel \ref{chapter:anforderungsanalyse} erhobenen Anforderungen betrachtet und die durchgeführte Usability-Evaluation beschrieben. \\
Zuletzt folgt noch das Kapitel \ref{chapter:fazit}, welches eine inhaltliche Zusammenfassung über die Ergebnisse dieser Arbeit und einen Ausblick auf weiterführende Ansätze gibt. Des weiteren wird auch eine kritische Reflexion des Verlaufs und der Ergebnisse dieser Arbeit abgeben.


