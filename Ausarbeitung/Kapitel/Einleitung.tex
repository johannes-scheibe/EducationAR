\chapter{Einleitung}\label{chapter:einleitung}
Das folgende Kapitel bildet die Grundlage dieser Arbeit. Es beleuchtet die Thematik und die daraus resultierende Forschungsfrage. Des weiteren werden die Ziele, sowie das Vorgehen der Arbeit definiert. Abschließend folgt ein Überblick über den Aufbau der Arbeit.

\section{Motivation}\label{sec:motivation}
In der heutigen Zeit werden immer neuere Methoden zur Informationsbeschaffung und -darstellung entwickelt, bei denen sich oft die Frage stellt, welche praktische Relevanz diese Verfahren besitzen. \\
Augmented Reality stellt eine dieser Technologien dar, die momentan immer mehr an Relevanz gewinnt. Sie bildet den Oberbegriff für die Erweiterung der Realität mit virtuellen Objekten (vgl. Kapitel \ref{sec:ar}). \\
Eines der wohl verbreitetsten Anwendungsgebiete stellt dabei der Entertainmentbereich dar. Hier wird die Technologie vor allem im Bereich der Spielentwicklung für Mobilgeräte genutzt. Die grundlegenden Technologien und Methoden lassen sich jedoch auch auf andere Bereiche übertragen. \\
Im (Hoch-)Schulumfeld könnte Augmented Reality eine neue Möglichkeit zur Visualisierung von Lerninhalten bieten, deren Vorteil darin liegt, zweidimensionale Abbildungen durch dreidimensionale Modelle zu ersetzen. \\
Auf dieser Grundlage wurde die folgende Forschungsfrage aufgestellt:\\
\glqq Wie lassen sich die Konzepte der Augmented Reality für den Bildungsbereich nutzen?\grqq \\
Dabei beschäftigt sich diese Arbeit sowohl mit konkreten Anwendungsfällen von Augmented Reality, als auch mit der Thematik, wie sich die Technologien aus den bereits sehr verbreiteten Einsatzgebieten für den Bildungsbereich nutzen lassen können. \\


\section{Zielsetzung der Arbeit}\label{sec:ziele}
Das Ziel dieser Arbeit ist die Erhebung von verschiedenen Anwendungsmöglichkeiten der Augmented Reality Konzepte für den Bildungsbereich. Dazu müssen bestehende Technologien und Methoden erfasst werden, aus denen dann gegebenenfalls im Anschluss neue Konzepte abgeleitet werden. \\
Auf dieser Grundlage soll dann eine solche Anwendung entworfen und die Realisierbarkeit durch eine prototypische Umsetzung validiert werden. \todo{ausformulieren}

\section{Vorgehensweise}

\section{Aufbau der Arbeit}
Insgesamt besteht diese Arbeit aus 8 Kapiteln.\\
Zu Beginn dieser Arbeit wird in dem Kapitel \ref{chapter:grundlagen}, die für das Verständnis notwendig sind, die Basis für das Verständnis dieser Arbeit beleuchtet. Dabei wird auf die grundlegenden Prinzipien von OpenGl, der Augemnted Reality und der Android App Entwicklung eingegangen.\\
Anschließend folgt das Kapitel \ref{chapter:arbeiten} in dem verschiedene Veröffentlichungen zum Thema dieser Arbeit beschrieben werden.\\
Diese beiden Kapitel bilden die Grundlage für die folgende Entwicklung des Prototyps. Im Kapitel \ref{chapter:anforderungsanalyse} werden die Anforderungen für den erhoben.\\
Im folgenden Kapitel \ref{chapter:entwurf} werden aus den Anforderungen erste Designentscheidungen abgeleitet und ein Entwurf für die Implementierung entwickelt.\\
Die Konkrete Implementierung wird in Kapitel \ref{chapter:implementierung} behandelt. Dabei werden die einzelnen, konkreten Klassen, deren Methoden, sowie die Beziehungen zwischen den Klassen beschrieben. \\
Im Kapitel \ref{chapter:evaluation} folgt eine Evaluation des Prototyps und des Einsatzes von Augmented Reality in der Bildung. Dazu werden zunächst die in Kapitel \ref{chapter:anforderungsanalyse} erhobenen Anforderungen betrachtet und die Realisierung durch den Prototypen geprüft. \\
Das abschließende Kapitel \ref{chapter:fazit} zieht ein Fazit die Ergebnisse dieser Arbeit gezogen und ein Ausblick gegeben. 
\todo{Beschreibung des letzten Kapitels überarbeiten}

